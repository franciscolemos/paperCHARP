Although in the airline industry there is a realization that the choice of cruise speed has a critical impact on the trade-off between reducing delays versus reducing fuel cost, the airlines do not take full advantage of this alternative.
%%%%%%%%%%%%%%%% AKTURK %%%%%%%%%%%%%%%%%%%%%%%%%%%%%%%%%%%
In the work of \citep{Aktuerk2014} the authors formulated the problem to recover a flight schedule, subject to a single (or multiple) disruption(s), while minimizing the sum of tardiness cost for all flight, the cost which is incurred if two aircraft are swapped and if they end up at different airports than originally planned in the initial schedule, the additional fuel and carbon emission cost due to increased cruise speed in the new schedule.
To repair the disrupted airline schedule the authors use three approaches, delay propagation, cruise speed control and the third approach is swap and cruise speed control. The problem is solve using conic mixed integer programming and the authors claimed at the time that this was the first implementation of a conic quadratic optimization approach to solve an aircraft recovery problem in an optimal manner.\\
%%%%%%%%%%%%%%%%%%%%%%%%%%%%%%%%%%%%%%%%%%%%%%%%%%%%%%%%%%%
%%%%%%%%%%%%%%%% ARIKAN %%%%%%%%%%%%%%%%%%%%%%%%%%%%%%%%%%%
The work of \citep{Arikan2016} uses a mathematical formulation for the integrated aircraft and passenger recovery problem that considers aircraft and passenger related costs simultaneously. The method uses a realistic fuel cost function, which was developed respecting the technical report of Airbus 2004.  The authors mention that airlines tend to operate their flights at maximum range cruise (MRC)  speeds to save fuel and that the relation of fuel cost with deviation from MRC speed is increasing and convex. The authors formulated the problem using conic quadratic mixed integer programming model and claim that the computational experiments can handle several simultaneous disruptions optimally on a four-hub network of a major U.S. airline within less than a minute on the average. They conclude that proposed approach is able to find optimal trade-off between cruise time controllability and passenger-related costs in real time.\\
%%%%%%%%%%%%%%%%%%%%%%%%%%%%%%%%%%%%%%%%%%%%%%%%%%%%%%%%%%%
%%%%%%%%%%%%%%%% Marla %%%%%%%%%%%%%%%%%%%%%%%%%%%%%%%%%%%%
In more recent work \citep{Marla2017} the authors introduce flight planning to explore the trade-off between delays and fuel burn. Flight planning consists of process to specify the route of a flight, its speed and fuel burn. This work makes use of the cost index (CI) for a flight to capture both the flight time and fuel burn.  The CI is the ratio of the time-related cost of an aircraft operation and the cost of fuel. The value of the CI reflects the relative effects of fuel cost on overall trip cost as compared to time-related direct operating costs.The authors evaluated, for each possible flight speed , the fuel cost and the passenger-related delay costs to the airline, using JetPlan, a flight planning tool developed by Jeppesen Commercial and Military Aviation and the impacts on passengers using an airline disruption management simulator. At CI 500, there is a sharp reduction in the passenger cost function as several passengers can make their planned connections, while at lower CI values these passengers misconnected. However the authors mention that the specific airline  from which the example was extracted, typically operates at CI 30 and allows its dispatchers and pilots to speed up to a maximum of CI 300. This standard of operation would not have any significant impact in minimizing the total costs, however it leads to the conclusion that it is possible to optimize total costs by increasing aircraft speeds relative to those initially planned.\\
